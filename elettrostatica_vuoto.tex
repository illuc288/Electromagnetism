\section{Elettrostatica nel vuoto} % (fold)
\label{sec:elettrostatica_nel_vuoto}

\subsection{Formule Sperimentali} % (fold)
\label{sub:formule_sperimentali_e}

\begin{equation}
    \brcurs = \mbf r - \mbf r'
\end{equation}
Dove r è il punto in cui vogliamo calcolare il campo e r' è la posizione della carica. (Se pensiamo alla forza, r è la posizione della carica su cui calcoliamo la forza, r' è l'altra).

\begin{equation}
    \mbf{F} = \frac{1}{4 \pi \epsilon_0 } \frac{q_1 q_2}{\rcurs^2} \hrcurs
\end{equation}

\begin{equation}
    \mbf{E} = \lim_{q \to 0} \frac{\mbf F}{q}
\end{equation}
Limite non matematico ma fisico, pensiamo di avere una carica di prova molto piccola in modo che non alteri le cariche che generano il campo.

\begin{equation}
    \mbf{E} = \frac{1}{4 \pi \epsilon_0 } \frac{q}{\rcurs^2} \hrcurs
\end{equation}

Principio di sovrapposizione:
\begin{equation}
    \mbf E_{tot} = \mbf E_1 + \mbf E_2 + \dots
\end{equation}

Distribuzioni di carica:

\begin{equation} \label{eq:dizionario_carica} 
    dq = \lambda dl = \sigma dS = \rho d\tau
\end{equation}

Dal principio di sovrapposizione:
\begin{equation} \label{eq:E_da_distrib} 
    \mbf{E} = \frac{1}{4 \pi \epsilon_0 } \int          \frac{dq}           {\rcurs^2} \hrcurs
            = \frac{1}{4 \pi \epsilon_0 } \int_\gamma   \frac{\lambda dl}   {\rcurs^2} \hrcurs
            = \frac{1}{4 \pi \epsilon_0 } \int_S        \frac{\sigma dS}    {\rcurs^2} \hrcurs
            = \frac{1}{4 \pi \epsilon_0 } \int_V        \frac{\rho d\tau}   {\rcurs^2} \hrcurs
\end{equation}

% subsection formule_sperimentali_e (end)

\subsection{Teorema di Gauss e prima equazione di Maxwell nel vuoto} % (fold)
\label{sub:teorema_di_gauss_e_prima_equazione_di_maxwell}

% subsection teorema_di_gauss_e_prima_equazione_di_maxwell (end)

Teorema di Gauss:
\begin{equation} \label{eq:gauss_int} 
    \Phi ( \mbf E ) = \int_S \mbf{ E \cdot \hat n} \, dS = \frac{ Q_{int} }{ \epsilon_0 } 
\end{equation}

\begin{proof}
    Pensiamo ad una singola carica $q$ posta nell'origine. Questo non lede alla generalità poiché se non è nell'origine possiamo traslarla e se abbiamo più cariche vale il principio di sovrapposizione e quindi il flusso totale è la somma dei singoli flussi. Abbiamo quindi:

    \begin{equation*}
        \Phi ( \mbf E ) = \frac{q}{4 \pi \epsilon_0 } \int_S \frac{ \vers r  \cdot \vers n dS }{ r^2 }
                        = \frac{q}{4 \pi \epsilon_0 } \int_S \frac{ dS_r }{ r^2 } 
                        = \frac{q}{4 \pi \epsilon_0 } \int d\Omega
                        = \frac{q}{\epsilon_0 }
    \end{equation*}

\end{proof}

Per scrivere il teorema di Gauss in forma differenziale vogliamo mostrare che vale l'equazione \ref{eq:div_def}
\begin{proof}
    Pensiamo ad un cubetto infinitesimo con assi paralleli agli assi cartesiani. Esso ha un vertice in $(x, y, z)$ e il vertice opposto in $(x + dx, y + dy, z + dz)$. Pensiamo al flusso attraverso le due facce ortogonali all'asse x:
    \begin{equation*}
        d\Phi_x = E_x(x + dx, y, z) dydz - E_x(x, y, z) dydz = \partial_x E_x dxdydz = \partial_x E_x dV
    \end{equation*}
    E varranno formule analoghe per le altre facce. Il flusso totale è quindi:
    \begin{equation*}
        d\Phi = (\partial_x E_x + \partial_y E_y + \partial_z E_z) dV = (\nabla \cdot \mbf E ) dV
    \end{equation*}
    quindi si ha:
    \begin{equation*}
        \nabla \cdot \mbf E 
        = \lim_{ dV \to 0 } \frac{ d\Phi }{ dV }
    \end{equation*}
    che è la tesi.
\end{proof}

Applicando questo teorema alla \ref{eq:gauss_int} si ha:
\begin{equation}
    \int_{\partial V} \mbf E \cdot d \mbf S = \int_V \nabla \cdot \mbf E dV = \int_V \frac{ \rho }{ \epsilon_0 } 
\end{equation}
Ma il volume di integrazione è del tutto arbitrario, per cui si ha:
\begin{equation} \label{eq:gauss_diff} 
    \nabla \cdot \mbf E = \frac{ \rho }{ \epsilon_0 } 
\end{equation}
Che è il teorema di Gauss in forma differenziale ovvero la prima equazione di Maxwell nel vuoto. 

\subsection{Seconda equazione di Maxwell nel vuoto} % (fold)
\label{sub:seconda_equazione_di_maxwell_nel_vuoto}

Pensando al campo generato da una carica puntiforme si può vedere che il campo elettrico è conservativo, in quanto si trova:
\begin{equation}
    \int_A^\mbf B \mbf E \cdot d \mbf l 
        = \qpe \int_A^B \frac{ Q }{ r^2 } \vers r \cdot d \mbf l 
        = \qpe[Q] \int_A^B \frac{ 1 }{ r^2 } dr
        = \qpe[Q] \left ( \frac{ 1 }{ r_A } - \frac{ 1 }{ r_B } \right )
        = V(A) - V(B)
\end{equation}
Ovvero l'integrale di linea dipende solo dagli estremi e si può scrivere:
\begin{equation}
    \mbf E = - \nabla V
\end{equation}
Ma ricordando che il rotore del gradiente è nullo si ha:
\begin{equation} \label{eq:rot_E_stat} 
    \nabla \times \mbf E = 0
\end{equation}

Un modo per mostrare che il rotore del gradiente è nullo è di applicare il teorema di Stokes ad una superficie qualsiasi:
\begin{equation}
    \int_S ( \nabla \times \nabla f ) \cdot d \mbf S = \int_{\partial S} \nabla f \cdot d \mbf l = f(A) - f(A) = 0
\end{equation}
Per l'arbitrarietà di $S$ si ha quindi che
\begin{equation}
    \nabla \times \nabla f = 0
\end{equation}

DIM STOKES BALZATA

Inoltre dall'equazione sopra si vede che il potenziale nel caso di una carica puntiforme nel punto $\mbf r'$ è dato dalla seguente formula:
\begin{equation}
    V(\mbf r) = \qpe[Q] \frac{ 1 }{ \rcurs } 
\end{equation}
Questa formula può poi essere generalizzata in modo analogo alla \ref{eq:E_da_distrib}:
\begin{equation} \label{eq:V_da_distrib} 
    V(\mbf r)   = \qpe \int \frac{ dq }{ \rcurs }
                = \qpe \int_\gamma \frac{ \lambda dl }{ \rcurs }
                = \qpe \int_S \frac{ \sigma dS }{ \rcurs }
                = \qpe \int_V \frac{ \rho d\tau }{ \rcurs }
\end{equation}

C'è poi un modo istruttivo di trovare l'ultima di queste uguaglianze. Partiamo dal campo elettrico
\begin{equation}
    \mbf E  = \qpe \int_V \frac{\rho \, d\tau} {\rcurs^2} \hrcurs
\end{equation}
e sfruttiamo come altre volte il fatto che 
\begin{equation}
    \frac{ \hrcurs }{ \rcurs^2 } = -\grad \left ( \frac{ 1 }{ \rcurs }  \right )
\end{equation}
Quindi:
\begin{equation}
    \mbf E  = -\qpe \int_V \rho \, d\tau \grad \left ( \frac{ 1 }{ \rcurs }  \right )
            = -\grad \left ( \qpe \int_V \frac{ \rho }{ \rcurs } d\tau \right )
            = -\grad V
\end{equation}
Abbiamo quindi ritrovato il potenziale di una distribuzione volumica.

% subsection seconda_equazione_di_maxwell_nel_vuoto (end)

\subsection{Dipolo} % (fold)
\label{sub:dipolo}

Carica $+q$ e carica $-q$ a distanza $\delta$. Vettore $\pmb \delta$ dalla carica negativa a quella positiva. $\mbf p$ momento di dipolo.
\begin{equation}
    \mbf p = q \pmb \delta
\end{equation}

Calcoliamo il potenziale ad $r >> \delta$. $r_+$ distanza dalla carica positiva, $r_-$ distanza dalla carica negativa. 
\begin{equation}
    V(\mbf r) = \qpe[q] \left ( \frac{ 1 }{ r_+ } - \frac{ 1 }{ r_- }  \right )
\end{equation}
Usando il teorema del coseno ed espandendo in serie di Taylor: ($\alpha$ è l'angolo tra il vettore posizione $\mbf r$ e $\mbf p$, dove r è preso dal centro del dipolo)
\begin{equation}
    \frac{ 1 }{ r_\pm } 
        = \frac{ 1 }{ r \sqrt{ 1 + (\delta/2r)^2 \mp \frac{\delta}{r} \cos \alpha} } 
        \sim \frac{ 1 }{ r \sqrt{ 1 \mp \frac{\delta}{r} \cos \alpha} } 
        \sim \frac{ 1 \pm \frac{\delta}{2r} \cos \alpha }{ r } 
\end{equation}
Quindi si ha:
\begin{equation} \label{eq:campo_dipolo} 
    V(\mbf r) 
        = \qpe[q] \frac{ \delta \cos \alpha }{ r^2 } 
        = \qpe[q \delta \cos \alpha] \frac{ 1 }{ r^2 } 
        = \qpe[\mbf p \cdot \vers r] \frac{ 1 }{ r^2 } 
\end{equation}

Applicando il gradiente a questa espressione si può poi trovare il campo elettrico. Si ha 
\begin{align*}
    \nabla \left ( \frac{ \mbf p \cdot \mbf r }{ r^3 } \right ) 
    &= \frac{ \nabla(\mbf p \cdot \mbf r) }{ r^3 } + \mbf p \cdot \mbf r \nabla \left ( \frac{ 1 }{ r^3 } \right ) \\
    \nabla (\mbf p \cdot \mbf r) &= \mbf p \\
    \nabla \left ( \frac{ 1 }{ r^3 } \right ) &= -3 \frac{ \vers r }{ r^4 } \\
    \nabla \left ( \frac{ \mbf p \cdot \mbf r }{ r^3 } \right )  &= \frac{ \mbf p }{ r^3 } - \frac{ 3 ( \mbf p \cdot \vers r) \vers r}{ r^3 }
\end{align*}
Da cui si ha 
\begin{equation}
    \mbf E(\mbf r) = -\nabla V = \qpe \frac{ 1 }{ r^3 } ( 3 ( \mbf p \cdot \vers r) \vers r  -  \mbf p )
\end{equation}

Si può poi mostrare che un dipolo in un campo elettrico $\mbf E$ è sottoposto alla seguente forza:
\begin{equation}
    \mbf F = (\mbf p \cdot \nabla ) \mbf E = \nabla ( \mbf{ p \cdot E })
\end{equation}
Dove la seconda formula vale solo nel caso in cui $\mbf p$ sia indipentende dalla posizione. Cosa che non sarà vera nei dielettrici.
Ha la seguente energia potenziale:
\begin{equation}
    U = - \mbf{p \cdot E}
\end{equation}
Subisce il seguente momento torcente:
\begin{equation}
    \mbf{ M = p \times E}
\end{equation}

DIM BALZATE

% subsection dipolo (end)

\subsection{Espansione multipoli} % (fold)
\label{sub:espansione_multipoli}
Riprendiamo la \ref{eq:V_da_distrib}.
\begin{equation}
    V(\mbf r) = \qpe \int_V \frac{ \rho d\tau }{ \rcurs }
\end{equation}
Cerchiamo ora di scrivere questa equazione in una forma più significativa. Stiamo pesando al caso in cui la distribuzione di carica sia localizzata in un certo volume limitato e vogliamo valutare il campo da questa generato a grande distanza.
Per fare questo espandiamo la funzione $f(\mbf r') = \frac{ 1 }{ \| \mbf {r - r'} \| } $ intorno a $\mbf r' = 0$:
\begin{equation}
    f(\mbf r') = f(\mbf r' = 0) + \nabla'f |_{_{\mbf r'= 0}} \cdot \mbf r' + \dots
\end{equation}
Dalla \ref{eq:derivate_utili_grad} abbiamo che:
\begin{equation}
    \nabla' f = \nabla' \frac{ 1 }{ \rcurs } = -\nabla \frac{ 1 }{ \rcurs } = \frac{ \hrcurs }{ \rcurs^2 } = \frac{ \vers r }{ r^2 } 
\end{equation}
In cui nell'ultima uguaglianza è stato posto $\mbf r' = 0$. Per cui l'espansione di $f$ è data da:
\begin{equation}
    f(\mbf r') = \frac{ 1 }{ r } + \frac{ \vers r }{ r^2 } \cdot \mbf r' + \dots
\end{equation}
Il potenziale può quindi esser scritto nel seguente modo:
\begin{equation}
    V(\mbf r)   = \qpe \int_V \rho \left ( \frac{ 1 }{ r } + \frac{ \vers r }{ r^2 } \cdot \mbf r' + \dots \right )
                = \qpe \left [ \frac{ 1 }{ r } \int_V \rho + \frac{ \vers r }{ r^2 } \cdot \int_V \rho \, \mbf r' + \dots \right ]
                = \qpe \left [ \frac{ 1 }{ r } Q + \frac{ \vers r }{ r^2 } \cdot \mbf p + \dots \right ]
\end{equation}
Dove naturalmente si ha:
\begin{align}
    Q &= \int_V \rho d \tau' \\
    \mbf p &= \int_V \rho \mbf r' d \tau'
\end{align}
Qui $\mbf p$ è la generalizzazione della quantità introdotta nel paragrafo precedente. Per ricondursi a quel caso specifico bisogna pensare ad una $\rho$ che abbia una delta di Dirac in corrispondenza delle cariche puntiformi.


% subsection espansione_multipoli (end)

\subsection{Conduttori} % (fold)
\label{sub:conduttori}
I conduttori hanno le seguenti proprietà:
\begin{enumerate}
    \item All'interno $\mbf E=0$
    \item Vicino alla superficie il campo elettrico è ortogonale alla superficie
    \item Il volume è tutto allo stesso potenziale
    \item Non vi è carica all'interno, ma è interamente distribuita sulla superficie
    \item Vicino alla superficie si ha $\mbf E = \sigma / \epsilon_0$
\end{enumerate}

Poiché ogni conduttore è un volume equipotenziale, ha senso chiedersi quanto sia la differenza di potenziale $V$ tra due conduttori. Pensiamo al caso in cui sul primo conduttore venga posta una carica $+Q$ e sull'altro una carica $-Q$. Allora il campo elettrico tra i due è dovuto ai soli conduttori ed è proporzionale alla carica $Q$, ma allora anche $V$ è proporzionale a $Q$. Chiamiamo capacità la costante di proporzionalità:
\begin{equation}
    C = \frac{ Q }{ V } 
\end{equation}


% subsection conduttori (end)

\subsection{Energia e pressione} % (fold)
\label{sub:energia_e_pressione}
Pensiamo all'energia di una configurazione di cariche. Per calcolarla possiamo pensare al lavoro esterno fatto su ogni singola carica per portarla nella sua posizione dall'infinito. Chiamiamo $q_i$ le cariche. Per posizionare in $\mbf r_1$la carica $q_1$ il lavoro è nullo poiché se non c'è ancora nessuna carica il campo è nullo. Per portare la carica $q_2$ dall'infinito a $\mbf r_2$ il lavoro svolto è dato da:
\begin{equation}
    L = \qpe[q_1q_2] \frac{ 1 }{ \rcurs_{12} } 
\end{equation}
Per portare una terza carica si ha poi:
\begin{equation}
    L = \qpe[q_1q_3] \frac{ 1 }{ \rcurs_{13} } + \qpe[q_2q_3] \frac{ 1 }{ \rcurs_{23} } 
\end{equation}

Quindi per $N$ cariche avremo:
\begin{equation}
    U = \sum_{i<j} \qpe[q_iq_j] \frac{ 1 }{ \rcurs_{ij} } 
\end{equation}
Che può anche esser scritta come 
\begin{equation}
    U = \frac{ 1 }{ 2 } \sum_{i \neq j} \qpe[q_iq_j] \frac{ 1 }{ \rcurs_{ij} } 
\end{equation}

Tuttavia qui possiamo riconoscere il campo di una carica puntiforme, per cui si può scrivere:
\begin{equation}
    U = \frac{ 1 }{ 2 } \sum_i q_i V_i
\end{equation}
Dove $V_i$ è il campo in posizione $\mbf r_i$ dovuto a tutte le ALTRE cariche.

Ma allora per una distribuzione continua si avrà:
\begin{equation} \label{eq:energia_elettrostatica_rhoV}
    U = \frac{ 1 }{ 2 } \int_V \rho V d\tau
\end{equation}
e altre formule analoghe per distribuzioni superficiali o lineari.

Da questa possiamo poi ricavare una formula generale per l'energia associata ad un campo elettrico. Ricordiamo infatti che $\div \mbf E = \rho / \epsilon_0 $, quindi:
\begin{equation}
    U   = \frac{ \epsilon_0 }{ 2 } \int_V \div \mbf E V d\tau 
        = \frac{ \epsilon_0 }{ 2 } \int_V \div (V\mbf E) - \int_V \mbf E \cdot (\grad V)
\end{equation}
Dove nella seconda uguaglianza abbiamo usato:
\begin{equation} 
    V (\div \mbf E) = \div (V\mbf E) - \mbf E \cdot (\grad V)
\end{equation}
che è semplicemente la \ref{eq:div_prod}.
Possiamo poi sfruttare il fatto che $\mbf E = -\grad V$ e il teorema della divergenza:
\begin{equation} \label{eq:energia_elettrostatica_finita} 
    U = \frac{ \epsilon_0 }{ 2 } \left ( \int_S (V\mbf E) \cdot d\mbf S + \int_V E^2 d\tau \right )
\end{equation}
Originariamente il dominio di integrazione era la regione che conteneva la carica $\rho$. Tuttavia è possibile espandere la regione di integrazione con $\rho = 0$ al di fuori del dominio originale. Espandendo il dominio la quantità a destra continua ad aumentare, mentre la parte a sinistra tende a zero, poiché il prodotto tra $V$ ed $\mbf E$ andrà come $1/r^3$, mentre il contributo dovuto all'integrale di superficie a come $r^2$. Quindi passando all'integrale su tutto lo spazio si ha:
\begin{equation} \label{eq:energia_elettrostatica} 
    U = \frac{ \epsilon_0 }{ 2 } \int_V E^2 d\tau
\end{equation}
Possiamo quindi dire che in tutto lo spazio vi è una densità di energia elettrica:
\begin{equation}
    u = \frac{ \epsilon_0 }{ 2 } E^2
\end{equation}

Pensando ad un conduttore, si vede che le cariche sulla superficie sentono una spinta verso l'esterno del conduttore. Infatti sulla superficie il campo complessivo è ancora zero, ma se pensiamo ad un quadratino sulla superficie, anche questo contribuisce al campo elettrico, tuttavia non può subire il suo stesso campo. è quindi sottoposto ad un campo che è dovuto al contributo del resto del conduttore. Per calcolarlo osserviamo che appena fuori si ha:
\begin{equation}
    \mbf E_{quadr} + \mbf E_{resto} = \frac{ \sigma }{ \epsilon_0 } 
\end{equation}
Ma $\mbf E_{quadr} = \sigma /2\epsilon_0$ poiché possiamo approssimarlo ad una superficie infinita. Quindi il campo a cui è sottoposto è $E_{resto} = \sigma /2\epsilon_0$. Si ha quindi:
\begin{align}
    F &= qE = \sigma dS \frac{ \sigma }{ 2 \epsilon_0 } \\
    p &= \frac{ F }{ dS } = \frac{ \sigma^2 }{ 2 \epsilon_0 } = \frac{ (\epsilon_0 E)^2 }{ 2 \epsilon_0 } = u
\end{align}
Ovvero la pressione subita dalle cariche sulla superficie è pari alla densità di energia elettrostatica appena fuori dal conduttore.

% subsection energia_e_pressione (end)

\subsection{Equazione di Poisson} % (fold)
\label{sub:equazione_di_poisson}
Con il potenziale la prima equazione di Maxwell può esser scritta come:
\begin{equation}
    \div \grad V = \nabla^2 V = - \frac{ \rho }{ \ez } 
\end{equation}
Questa è l'equazione di Poisson. Tuttavia se noi risolviamo questa troviamo $V$ e prendendo il gradiente troviamo $\mbf E$. Ma quindi la seconda equazione di Maxwell era inutile? NO, è proprio grazie a quella che possiamo scrivere $\mbf E = - \grad V$. 

L'equazione di Poisson da sola ha infinite soluzioni. Tuttavia con opportune condizioni al contorno si ha invece una soluzione unica. Possiamo ad esempio prendere come condizione al contorno il valore di $V$ sul bordo del dominio considerato (ad esempio che $V$ vada a 0 all'infinito). Mostriamo che in questo caso la soluzione è unica.
\begin{proof}
    Supponiamo per assurdo che ci siano due soluzioni, ovvero due funzioni $f_1$ e $f_2$ che entrambe soddisfano l'equazione di Poisson e hanno lo stesso valore sul bordo S. Consideriamo allora la funzione $f = f_1-f_2$. Si ha:
    \begin{equation}
        \nabla^2 f = \nabla^2 f_1 - \nabla^2 f_2 = - \frac{ \rho }{ \ez } + \frac{ \rho }{ \ez } = 0
    \end{equation}
    Inoltre $f=0$ su S. Mostriamo che allora dev'essere $f=0$ ovunque. Consideriamo l'espressione: $\div (f \grad f)$. Questa è utile poiché può essere integrata in due modi. Usando il teorema della divergenza: 
    \begin{equation}
        \int_V \div (f \grad f) = \int_S f \grad f = 0
    \end{equation}
    poiché $f=0$ su $S$. Oppure integrando per parti:
    \begin{align}
        \div (f \grad f) &= (\grad f)^2 + f \nabla^2 f \\
        \int_V \div (f \grad f) &= \int_V (\grad f)^2 + \int_V f \nabla^2 f = \int_V (\grad f)^2 
    \end{align}
    Ma come abbiamo visto con l'altro metodo questo deve fare 0. Poiché l'integranda è non negativa, l'unico modo in cui l'integrale può essere 0 è che $\grad f$ sia 0 ovunque. Ma allora $f$ è costante sul dominio, ma poiché vale 0 sul bordo, deve valere 0 ovunque. Di conseguenza $f_1=f_2$.
\end{proof}
Come conseguenza di questo, l'equazione di Laplace:
\begin{equation}
    \nabla^2 V = 0
\end{equation}
ha anch'essa un'unica soluzione se il potenziale è definito sul bordo del dominio di integrazione. Infatti è un caso particolare di quanto appena mostrato, con $\rho = 0$. Quindi se cerchiamo la soluzione dell'equazione di Laplace in un dominio vuoto contenente anche $n$ conduttori, con potenziali $V_i$, allora la soluzione è unica. Questo è il problema di Dirichlet. Al contrario, se anziché essere fissati i potenziali sui conduttori, sono fissate le cariche su di esse, abbiamo il problema di Von Neumann che ha anch'esso una soluzione unica. 

Questo segue dall'unicità della soluzione del problema di Dirichlet, perché tra le cariche sui conduttori e i potenziali vale una relazione analoga a quella del paragrafo precedente:
\begin{equation}
    \begin{pmatrix}
        Q_1 \\
        \vdots \\
        Q_2
    \end{pmatrix}
    = C 
    \begin{pmatrix}
        V_1 \\
        \vdots \\
        V_2
    \end{pmatrix}
\end{equation}
Dove $C$ adesso è una matrice e non più solo un numero. Si può poi mostrare che $C$ è simmetrica e definita positiva, per cui è invertibile. Quindi noti i $V_i$ i $Q_i$ sono univocamente determinati e viceversa. 

Per le soluzioni dell'equazione di Laplace, dette funzioni armoniche, vale poi il teorema della media: il valor medio che la funzione assume su una sfera è pari al valore che assume nel centro.
\begin{proof}
    Abbiamo $\nabla^2 f = 0$. Definiamo:
    \begin{equation}
        \bar f (r) = \frac{ 1 }{ S } \int_S f(\mbf r) dS = \frac{ 1 }{ 4 \pi r^2 } \int_S f(\mbf r) dS
    \end{equation}
    Ma possiamo poi passare dall'integrale su S all'integrale sull'angolo solido, poiché $d\Omega = dS/r^2$. 
    \begin{equation}
        \bar f (r) = \frac{ 1 }{ 4 \pi } \int_S f(\mbf r) d\Omega
    \end{equation}
    Possiamo ora derivare in r per capire l'andamento di $\bar f(r)$.
    \begin{equation}
        \frac{ d }{ dr } \bar f (r) 
            = \frac{ 1 }{ 4 \pi } \int_S \frac{ d }{ dr } f(\mbf r) d\Omega 
            = \frac{ 1 }{ 4 \pi } \int_S \grad f \cdot d \mbf S
            = \frac{ 1 }{ 4 \pi } \int_V \div (\grad f) dV
            = \frac{ 1 }{ 4 \pi } \int_V \nabla^2 f dV = 0
    \end{equation}
    Ma allora $\bar f(r) = cost = \bar f(0) = f(0)$
\end{proof}
Come conseguenza diretta di questo, si ha che le funzioni armoniche non ammettono massimi o minimi, in quanto se ci fosse un massimo, questo dovrebbe essere la media dei valori su ogni sfera centrata in quel punto. Quindi o la funzione assume il valore massimo su tutte le sfere, per cui in realtà non è un massimo. Oppure su ogni sfera vengono assunti sia maggiori che minori di questo, ma anche questo è assurdo. 

Quanto detto in questo paragrafo giustifica il metodo delle cariche immagini. Se vogliamo risolvere il problema di Dirichlet e riusciamo a trovare una configurazione di cariche all'esterno del dominio di integrazione per cui le condizioni al contorno siano le stesse del problema, allora queste ci danno direttamente la soluzione che per quanto detto prima è unica. Ad esempio si può pensare ad un piano conduttore infinito con $V=0$e cercare la soluzione da un lato di questo, con una carica $+q$ ad una distanza $d$ dal piano. Osserviamo che una carica $-q$ a distanza $d$ dal piano, dal lato opposto della carica +, è tale che il potenziale dovuto a queste due cariche sia nullo sul piano. Quindi, dal lato della carica +, la configurazione è uguale identica a quella del problema, quindi il potenziale da questo lato può essere trovato banalmente sommando i potenziali delle due cariche.

% subsection equazione_di_poisson (end)

% section elettrostatica_nel_vuoto (end)